\documentclass[../main.tex]{subfiles}

\begin{document}
	\section{Accounting Equations}
	
	\subsection{Return on Assets (ROA)}
	
	Returns on assets is in ratio form as income divided by assets invested \ie 
	
	\[
	\text{ROA} = \frac{\text{Net Profit}}{\text{Average total assets}}
	\]
	
	\subsection{Debt Ratio}
	
	The debt to assets ratio helps evaluate the level of debt risk. We 
	determine a company’s ability to pay it’s debts (liabilities) using the 
	debit ratio. The debt ratio is equal to total liabilities divided by total 
	assets.  
	\[
	\text{Debt Ratio} = \frac{\text{Total Liabilities}}{\text{Total Assets}}
	\]
	A higher ratio indicates that there is greater probability a 
	company will not be able to pay it’s debts in the future.
	
	\subsection{Profit Margin}
	
	Profit margin tells us about the relationship between sales and net 
	profit. We calculate the ratio by dividing net profit for the period by 
	sales revenue.
	
	\[
	\text{Profit Margin} = \frac{\text{Net Profit}}{\text{Net Sales}}
	\]
	
	A high profit margin is an indicator of future growth.
	
	\subsubsection{Current Ratio}
	
	The current ratio of a company gives us a good indication of the company’s 
	ability to pay its debts when they fall due. The current ratio is 
	calculated by dividing current assets by current liabilities. 
	\[
	\text{Current Ratio} = \frac{\text { Current Assets }}{\text { Current 
	Liabilities }}
	\]
	
	\subsubsection{Days' Sales Uncollected}
	
	The Days’ Sales Uncollected ratio indicates how much time is likely to pass before we receive cash 
	receipts from credit sales. It is calculated as Accounts Receivable divided by Net Sales times 365 
	days.
	
	\[
	\text{Days' Sales Uncollected} = \frac{\text{Accounts Receivable}}{\text{Net Sales}} \times 365
	\]
\end{document}